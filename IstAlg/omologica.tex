\chapter{Algebra omologica}

\section{$E$-strutture e $H^1$ di gruppi}

\begin{lemma}
    Sia $E\subset F$ finita di Galois con gruppo $\Gamma$; $V$ un $E$-spazio vettoriale.
    \begin{enumerate}
        \item Detto $V_F=F\otimes_E V$, $\Gamma$ agisce su $V_F$ tramite $\gamma(\lambda\otimes v)=\gamma(\lambda)\otimes v$ per $v\in V$. Valgono allora
        \begin{itemize}
            \item $V_F^\Gamma = V$
            \item $\gamma(\lambda v)=\gamma(\lambda)\gamma(v)$ per $v\in V_F$
        \end{itemize}
        \item Se $W$ è un $F$ spazio vettoriale, e $\Gamma$ agisce su $W$ in modo che $\gamma(\lambda v)=\gamma(\lambda)\gamma(v)$, allora detto $W_0=W^\Gamma$ valgono
        \begin{itemize}
            \item $W_0$ è un $E$-spazio vettoriale
            \item la mappa $F\otimes_E W_0\to W$ che manda $\lambda\otimes v\mapsto \lambda v$ è un isomorfismo
        \end{itemize}
    \end{enumerate}
\end{lemma}


\begin{theorem}
    Sia $E\subset F$ un'estensione di Galois infinita. Allora $$\Gal{F}{E}\cong \lim_{\substack{\longleftarrow\\ [L:E]<\infty}}\Gal{L}{E}$$
\end{theorem}
\begin{definition}
    Data $E\subset F$ di Galois, un'azione di $\Gal{F}{E}$ su uno spazio $X$ è detta \emph{continua} se $\forall x\in X\;\exists L\supset E$ finita di Galois, con $\Gal{F}{L}\subset\stab x$
\end{definition}

\begin{lemma}
    Sia $E\subset F$ di Galois infinita, $V$ un $F$-spazio vettoriale; supponiamo che $\Gamma=\Gal{F}{E}$ agisca in modo continuo su $V$ e valga $\gamma(\lambda v)=\gamma(\lambda)\gamma(v)$; allora valgono le conclusioni del punto $2$ del lemma precedente, in particolare $V\cong F\otimes_E V^\Gamma$.
\end{lemma}

\begin{definition}
    Sia $E\subset F$ estensione di Galois con gruppo $\Gamma$; siano $A_0,B$ due $E$-algebre, e $R=F\otimes A_0\cong F\otimes B$. Sia $\gamma_0 = \gamma\otimes_E \id_{A_0}$ l'azione di $\Gamma$ estesa su $R$, e similmente $\gamma_B$.\\
    Sia $c:\Gamma\to \Aut(R)$ dato da $c_\gamma=\gamma_B\cdot\gamma_A^{-1}$.
\end{definition}

\begin{proposition}$ $
    \begin{itemize}
        \item Per ogni $\gamma$, $c_\gamma$ è $F$-lineare, quindi ho definito $c:\Gamma\to \Aut_F(R)$
        \item Vale $c_{\gamma\delta}=c_\gamma\circ ^\gamma c_\delta$, dove $^\gamma\varphi=\gamma_0\circ\varphi\circ\gamma_0^{-1}$
    \end{itemize}
\end{proposition}

\begin{definition}
    Definiamo $Z^1(\Gamma, \Aut_F(R)) = \{ c:\Gamma\to\Aut_F(R)\st c_{\gamma\delta}=c_\gamma\circ ^\gamma c_\delta \}$.\\
    Se $c,c'\in Z^1$, diciamo che $c\sim c'$ se $\exists f\in\Aut_F R$ tale che $c'_\gamma=f\circ c_\gamma\circ ^\gamma f^{-1}$.\\
    Definiamo poi $H^1(\Gamma, \Aut_F(R))=\faktor{Z^1(\Gamma, \Aut_F(R))}{\sim}$
\end{definition}

\begin{theorem}
    $$\faktor{ \{ E\text{-strutture di }A \} }{\text{isomorfismo}} \cong H^1(\Gamma, \Aut_F(R)) $$
\end{theorem}


\bigskip
Sia ora $G$ un gruppo che agisce sullo spazio $X$; sia $\Gamma$ un gruppo che agisce su $G$ conservando il prodotto e su $X$ compatibilmente con $G$.

\begin{definition}
    Sia $Z^1(\Gamma,G)=\{ c:\Gamma\to G \st c_{\gamma\delta}=c_\gamma\cdot ^\gamma c_\delta \}$.\\
    Diciamo inoltre che $c\sim d$ se esiste $g\in G$ tale che $d_\gamma = g\cdot c_\gamma\cdot ^\gamma(g^{-1})$.\\
    Sia infine $H^1(\Gamma, G)=\faktor{Z^1(\Gamma,G)}{\sim}$.
\end{definition}

\begin{proposition}
    Sia $1\rightarrow H\rightarrow G\rightarrow K\rightarrow 1$ una successione esatta di gruppi su cui agisce $\Gamma$ in modo compatibile con le mappe.\\
    Allora $1\rightarrow H^\Gamma\rightarrow G^\Gamma\rightarrow K^\Gamma\rightarrow H^1(\Gamma, H)\rightarrow H^1(\Gamma, G)\rightarrow H^1(\Gamma, K)$ è una successione esatta di insiemi puntati.
\end{proposition}

\begin{theorem}[Hilbert 90]
    Sia $E\subset F$ di Galois finita con gruppo $\Gamma$; sia $G=\GL_n(F)$ e $\Gamma$ agisce su $G$ coefficiente per coefficiente. Allora $H^1(\Gamma,\GL_n(F))=\{1\}$.
\end{theorem}

\begin{corollary}
    Anche $H^1(\Gamma,\SL_n(F))=\{1\}$.
\end{corollary}

\begin{proposition}
    Sia $x_0\in X^\Gamma$ e $H=\stab_G x_0$; supponiamo che $G$ agisca transitivamente su $X$. Se $H^1(\Gamma, G)=\{1\}$, allora le orbite di $G^\Gamma$ in $X^\Gamma$ sono in bigezione con $H^1(\Gamma, H)$.
\end{proposition}



\section{Categorie, categorie abeliane e complessi}

\begin{lemma}[di Yoneda]
	Sia $ \mathcal{C} $ una categoria piccola e $ F \colon \mathcal C \to \mathbf{Set} $ un funtore controvariante nella categoria degli insiemi. Si ha una bigezione
	\[ \text{Nat}(h_X, \, F) \leftrightarrow F(X). \] 
\end{lemma}

\begin{proposition}[che fa impallidire anche Yoneda]
	I limiti sono coequalizzatori. Sia $ I $ un sistema proiettivo, allora $ \varprojlim F(i) $ è isomorfo all'equalizzatore del diagramma
	\[ \begin{tikzcd}
	\prod_{i} F(i)  \rar[shift left = 0.5ex, "T"] \rar[shift right = 0.5ex, "S" below] & \displaystyle\prod_{\varphi\colon i \to j} F(j)_\varphi
	\end{tikzcd} \]
	dove otteniamo $ T $ ed $ S $ sfruttando la proprietà universale el prodotto per completare i diagrammi
	\[ \begin{tikzcd}
	\prod_{i} F(i) \rar["T", dashed] \dar["\pi_b"] &
	\displaystyle\prod_{\varphi\colon i \to j} F(j)_\varphi \dar["\pi_{b\varphi}"] \\
	F(b) \rar["\id"] &
	F(b)_\varphi,
	\end{tikzcd} \qquad\qquad
	\begin{tikzcd}
	\prod_{i} F(i) \rar["S", dashed] \dar["\pi_a"] &
	\displaystyle\prod_{\varphi\colon i \to j} F(j)_\varphi \dar["\pi_{b\varphi}"] \\
	F(a) \rar["\varphi"] &
	F(b)_\varphi.
	\end{tikzcd}\]
\end{proposition}

\begin{definition}
    Una categoria $\mathcal{C}$ si dice \emph{additiva} se soddisfa le seguenti proprietà:
    \begin{itemize}
        \item $\forall X,Y\in\mathcal{C}$, l'insieme $\Hom(X,Y)$ è un gruppo abeliano
        \item La composizione di morfismi $\Hom(Y,Z)\times\Hom(X,Y)\to\Hom(X,Z)$ è bilineare
        \item Esiste un oggetto zero, cioè $0\in\mathcal C$ tale che $\Hom(X,0)=\Hom(0,X)$ sono il gruppo banale
        \item Dati $X,Y\in\mathcal C$ esiste il coprodotto $X\coprod Y$, definito dalla seguente proprietà universale:
        $\begin{tikzcd}
         & Z & \\
         X \ar[r,"i_X"'] \ar[ur, "f"] & X\coprod Y \ar[u,dashrightarrow] & Y \ar[l,"i_Y"] \ar[ul, "g"']
        \end{tikzcd}$
    \end{itemize}
\end{definition}

\begin{proposition}
    In una categoria additiva, il coprodotto è isomorfo al prodotto, e si indica con $X\oplus Y$.
\end{proposition}

\begin{definition}
    Fissata una mappa $\varphi: X\to Y$, diciamo che:
    \begin{itemize}
        \item $\ell:Z\to X$ è il \emph{nucleo} di $\varphi$ se $\varphi\circ\ell=0$ e per ogni $\alpha: U\to X$ tale che $\varphi\circ \alpha=0$, esiste un'unica $\tilde\alpha:U\to Z$ che faccia commutare\\
        $$ \begin{tikzcd}
        Z \ar[r, "\ell"] & X \ar[r, "\varphi"] & Y \\
         & U \ar[u,"\alpha"] \ar[ul, dashrightarrow, "\tilde\alpha"]
        \end{tikzcd} $$
        \item $m:Y\to Q$ è il \emph{conucleo} di $\varphi$ se $m\circ\varphi=0$ e per ogni $\beta: Y\to U$ tale che $\beta\circ\varphi=0$, esiste un'unica $\tilde\beta:Q\to U$ che faccia commutare\\
        $$ \begin{tikzcd}
        X \ar[r, "\varphi"] & Y \ar[r, "m"] \ar[d,"\beta"] & Q \ar[dl, dashrightarrow, "\tilde\beta"] \\
        & U 
        \end{tikzcd} $$
    \end{itemize}
\end{definition}

\begin{definition}
    Una categoria additiva $\mathcal{C}$ è detta \emph{abeliana} se per ogni mappa $\varphi: X\to Y$ esistono $\alpha=\ker\varphi$ e $\beta=\coker\varphi$, e inoltre $\coker\alpha\cong\ker\beta$ e questo oggetto si dice $\Imm\varphi$.
    $$\begin{tikzcd}
    K \ar[r,"\alpha"] & X \ar[r,"\varphi"] \ar[d, "\pi"] & Y \ar[r,"\beta"] & Q \\
     & \coker\alpha \ar[ur, dashrightarrow, "\psi"] \ar[r, dashrightarrow, "\Phi"',"\sim"] & \ker\beta \ar[u,"j"] &
    \end{tikzcd}$$
\end{definition}

\begin{proposition}
    In una categoria abeliana $\varphi$ è un isomorfismo se e solo se $\ker\varphi=0$ e $\coker\varphi=0$.
\end{proposition}

\begin{definition}
    La successione $X\xrightarrow{\varphi} Y \xrightarrow{\psi} Z$ si dice \emph{esatta} in $Y$ se $\psi\varphi=0$ e vale $\Imm\varphi\cong \ker\psi$ (o equivalentemente $\Imm\psi\cong\coker\varphi$).
\end{definition}

\begin{proposition}
    In una categoria abeliana valgono lo snake lemma e il lemma dei 5.
\end{proposition}

\begin{lemma}
	Nella categoria degli $ A $-moduli il limite inverso $ \varprojlim $ è esatto a sinistra.
\end{lemma}

\begin{theorem}
    In realtà le categorie abeliane sono gli $A$-moduli...
\end{theorem}

\subsection{Complessi}
Mettiamoci in una categoria abeliana $\mathcal{C}$.

\begin{definition}
    Una \emph{complesso} $X^\bullet$ è una successione di oggetti e frecce $$ \dots\xrightarrow{\de^{n-1}} X^n \xrightarrow{\de^{n}} X^{n+1} \xrightarrow{\de^{n+1}}\dots $$
    tali che $\de^{n+1}\circ\de^n = 0$ per ogni $n\in\Z$.\\
    Un \emph{morfismo di complessi} $\varphi^\bullet:X^\bullet\to Y^\bullet$ è una successione di mappe $\varphi^n:X^n\to Y^n$ tali che tutti i quadrati commutino:
    $$\begin{tikzcd}
    X^{n-1} \ar[r,"\de_X^{n-1}"] \ar[d,"\varphi^{n-1}"] & X^{n} \ar[r,"\de_X^{n}"] \ar[d,"\varphi^{n}"] & X^{n+1} \ar[d,"\varphi^{n+1}"] \\
    Y^{n-1} \ar[r,"\de_Y^{n-1}"] & Y^{n} \ar[r,"\de_Y^{n}"] & Y^{n+1}
    \end{tikzcd}$$
    Possiamo allora considerare le categorie $\Com(\mathcal{C})$, $\Com^+(\mathcal{C})$ e $\Com^-(\mathcal{C})$ dei complessi (eventualmente limitati in una delle due direzioni).
\end{definition}

\begin{proposition}
    Le categorie $\Com(\mathcal{C})$, $\Com^+(\mathcal{C})$ e $\Com^-(\mathcal{C})$ sono abeliane.
\end{proposition}

\begin{definition}
    Sia $X^\bullet$ un complesso; definiamo $Z^n(X)=\ker(\de^n)$ e $B^n(X)=\Imm(\de^{n-1})$. Definiamo poi $H^n(X)=\faktor{Z^n(X)}{B^n(X)}$ l'$n$-esimo gruppo di coomologia.
\end{definition}

\begin{proposition}
    Se $\varphi^\bullet:X^\bullet\to Y^\bullet$ è morfismo di complessi, otteniamo una successione di mappe $H^n(\varphi):H^n(X)\to H^n(Y)$.
\end{proposition}

\begin{theorem}
    Sia $0\rightarrow X^\bullet\xrightarrow{\varphi^\bullet} Y^\bullet \xrightarrow{\psi^\bullet} Z^\bullet\rightarrow 0$ una successione esatta di complessi. Allora la successione $$H^n(X)\to H^n(Y)\to H^n(Z)\xrightarrow{\omega_n} H^{n+1}(X)\to H^{n+1}(Y)\to H^{n+1}(Z)$$
    è esatta.
\end{theorem}

\begin{definition}
    Sia $\varphi: X^\bullet\to Y^\bullet$; diciamo che $\varphi\sim0$ è \emph{omotopa} a $0$ se esistono delle mappe $h^n: X^n\to Y^{n-1}$ tali che $\varphi^n=\de_Y^{n-1}\circ h^n + h^{n+1}\circ\de_X^n$.
\end{definition}

\begin{proposition}
    Se $\varphi\sim0$, allora vale $H^n(\varphi)=0$; in particolare se vale $\id\sim0$, allora il complesso è esatto.
\end{proposition}

\begin{definition}
    Se $X^\bullet$ è un complesso, diciamo che il complesso $Y^\bullet$ è una \emph{risoluzione iniettiva} di $X^\bullet$ se gli $Y^n$ sono oggetti iniettivi ed esiste un morfismo di complessi $\varphi^\bullet:X^\bullet\to Y^\bullet$ che sia un isomorfismo in coomologia.
\end{definition}

\begin{definition}
    Sia $A\in\mathcal C$ e $F:\mathcal C\to\mathcal C'$ un funtore additivo, esatto a sinistra.\\
    Sia $I^\bullet$ una risoluzione iniettiva del complesso $\dots\to0\to A\to0\to\dots$.\\
    Definiamo l'$i$-esimo funtore derivato come $R^iF(A)=H^i(FI^\bullet)$ l'$i$-esimo gruppo di coomologia del complesso $0\to FI^0\to FI^1\to\dots$.
\end{definition}
\begin{oss}
    Verificheremo che la risoluzione iniettiva esiste, e che il funtore derivato non dipende dalla scelta della risoluzione iniettiva.
\end{oss}


\begin{definition}
    Se $X^\bullet$ è un complesso, diciamo che il complesso $P^\bullet$ è una \emph{risoluzione proiettiva} di $X^\bullet$ se i $P^n$ sono oggetti iniettivi ed esiste un morfismo di complessi $\varphi^\bullet:P^\bullet\to X^\bullet$ che sia un isomorfismo in coomologia.
\end{definition}

\begin{definition}
    Sia $A\in\mathcal C$ e $F:\mathcal C\to\mathcal C'$ un funtore controvariante, additivo, esatto a sinistra.\\
    Sia $P^\bullet$ una risoluzione iniettiva del complesso $\dots\to0\to A\to0\to\dots$.\\
    Definiamo l'$i$-esimo funtore derivato come $L^iF(A)=H^i(FP^\bullet)$ l'$i$-esimo gruppo di coomologia del complesso $0\to FP^0\to FP^1\to\dots$.
\end{definition}

\begin{definition}
    Siano $X,Y$ oggetti; siano $F,G$ i funtori $F=\Hom(X,-)$ e $G=\Hom(-,Y)$.\\
    Definiamo allora $\Ext^i(X,Y)=R^iF(Y)$ e $\underline{\Ext}^i(X,Y)=L^iG(X)$.
\end{definition}

\begin{example}
    Siano $m,n\in\Z$ e scriviamo $m=dm',n=dn'$ con $d=\gcd(m,n)$.\\
    Allora $\underline{\Ext}^0(\faktor{\Z}{(m)}, \faktor{\Z}{(m)})\cong \underline{\Ext}^1(\faktor{\Z}{(m)}, \faktor{\Z}{(m)})\cong \faktor{\Z}{(d)}$.
\end{example}

\begin{proposition}
    Se $X,Y$ sono oggetti, allora $\Ext^i(X,Y)\cong\underline{\Ext}^i(X,Y)$.
\end{proposition}

\section{Coomologia di gruppi}

Sia $G$ un gruppo e $R$ un anello commutativo con unità. Lavoreremo nella categoria degli $R[G]$ moduli, dove $R[G]=\bigoplus_{g\in G}Re_g$.

\begin{definition}
    Sia $M$ un $R[G]$-modulo; definiamo $F_1(M)=M^G=\Hom_{R[G]}(R,M)$ e $F_2(M)=\faktor{M}{\langle m-gm \rangle}=R\otimes_{R[G]}M$.\\
    Definiamo $H^n(G,M)=\Ext^n(R,M)=R^nF_1(M)$; sappiamo però che è isomorfo a $\underline{\Ext}^n(R,M)$, che è il funtore derivato di $\Hom_{R[G]}(-,M)$.\\
    Inoltre $H_n(G,M)=\Tor_n(R,M)$ il funtore derivato di $F_2$.
\end{definition}

\begin{proposition}[risoluzione libera di $R$]
    Siano $P^0=R[G],P^{-1}=R[G\times G],P^{-2}=R[G\times G\times G],\dots$; sia $\eps:P^0\to R$ data da $\eps(g)=1$.\\
    Sia poi $\de^{-n}:P^{-n}\to P^{-n+1}$ data da $$\de^{-n}(e_{g_0,\dots,g_n})=\sum_{i=0}^n (-1)^i e_{g_0,\dots,\hat{g_i},\dots,g_n}$$
    Allora la successione $0\leftarrow R\xleftarrow{\eps} P^0\xleftarrow{\de^{-1}}P^{-1}\xleftarrow{\de^{-2}}\dots$ è una risoluzione proiettiva di $R$.
\end{proposition}

\begin{proposition}
    La mappa $\Phi_n:\Hom_{R[G]}(P^{-n},M)\to \{ f:G^n\to M \}=:C^n(G,M)$ data da $\Phi_n(\psi)(g_1,\dots,g_n)=\psi(1,g_1,g_1g_2,\dots,g_1\cdots g_n)$ è un isomorfismo.\\
    Inoltre la mappa $\delta_C^n:C^n(G,M)\to C^{n+1}(G,M)$ data da $\delta_C^n f=\Phi_{n+1}((\Phi_n^{-1}f)\circ\de^{-n})$ ha la formula esplicita \begin{dmath*} (\delta_C^n f)(g_1,\dots,g_{n+1}) = g_1\cdot f(g_2,\dots,g_{n+1})+\sum_{i=1}^n (-1)^i f(g_1,\dots, g_i\cdot g_{i+1},\dots, g_{n+1})+(-1)^nf(g_1,\dots,g_n) \end{dmath*}
\end{proposition}

\begin{example}
    Osserviamo che $(\delta^0f)(g)=gf(1)-f(1)$ e $(\delta^2f)(g,h)=gf(h)-f(gh)+f(g)$, ovvero $Z^1(G,M)=\{ f:G\to M\st f(gh)=f(g)+gf(h) \}$ e perciò $H^1(G,M)=\faktor{Z^1}{ \{g\mapsto gm-m \} }$, che è esattamente la definizione data nel capitolo precedente con i cocicli $c_\gamma$.
\end{example}

\begin{definition}
    Sia $f:H\to G$ un omomorfismo di gruppi e $M$ un $G$-modulo. Allora $f^\ast M$ è un $H$-modulo tramite l'azione $h\cdot m=f(h)m$.\\
    Inoltre $f$ induce un morfismo di complessi $C^q(G,M)\to C^q(H,M)$, da cui si ottiene una mappa $Res^q: H^q(G,M)\to H^q(H,f^\ast M)$ in coomologia. \\
\end{definition}

\begin{definition}
        Sia $f:H\to G$ un omomorfismo di gruppi e $N$ un $H$-modulo.\\
        Definiamo $\ind_H^G N=R[G]\otimes_{R[H]} N$ che è un $G$-modulo tramite $g\cdot(x\otimes n)=xg\otimes n$.\\
        Definiamo poi $\coind_H^G N=\Hom_H(R[G],N)$, dove $R[G]$ è un $H$-modulo tramite l'azione $h\cdot g= gf(h^{-1})$. Questo ha una struttura di $G$-modulo con l'azione $(g\cdot\varphi)(x)=\varphi(g^{-1}x)$.
\end{definition}

\begin{proposition}
    Sia $M$ un $G$-modulo e $N$ un $H$-modulo, e $f:H\to G$. Allora valgono
    \begin{itemize}
        \item $\Hom_H(N,f^\ast M)\cong\Hom_G(\ind_H^G,M)$
        \item $\Hom_H(f^\ast M, N)\cong\Hom_G(M,\coind_H^G N)$
    \end{itemize}
\end{proposition}

\begin{proposition}
    Siano $F,G$ due funtori aggiunti tra due categorie $\mathcal A,\mathcal B$, ovvero tali che $\Hom_{\mathcal A}(a,Gb)=\Hom_{\mathcal B}(Fa,b)$. Allora valgono:
    \begin{itemize}
        \item $F$ conserva i limiti diretti, è esatto a destra e manda proiettivi in proiettivi
        \item $G$ conserva i limiti inversi, è esatto a sinistra e manda iniettivi in iniettivi
        \item Se $\mathcal A,\mathcal B$ sono abeliane, $F$ e $G$ sono additivi
    \end{itemize}
\end{proposition}

\begin{theorem}
    Sia $H<G$ e $N$ un $H$-modulo. Allora
    \begin{itemize}
        \item $H^i(G,\coind_H^G N)=H^i(H,N)$
        \item $H_i(G,\ind_H^G N)=H_i(H,N)$
    \end{itemize}
\end{theorem}

\begin{definition}
    Dato $G$ un gruppo, $H<G$ di indice finito, $M$ un $G$-modulo, abbiamo le mappe:
    \begin{itemize}
        \item $i:M^G\to M^H$ la mappa di inclusione
        \item $N:M^H\to M^G$ la ``norma'': se $x_1,\dots,x_n$ sono i rappresentanti di $G/H$, definiamo $N(m)=\sum x_im$
    \end{itemize}
    Inoltre, se $0\to M\to I_M^\bullet$ è risoluzione iniettiva come $G$-modulo, lo è anche come $H$-modulo.
    
    Perciò le due mappe passano a mappe di complessi, e quindi in coomologia:
    \begin{itemize}
        \item $\res^q:H^q(G,M)\to H^q(H,M)$
        \item $\cores^q:H^q(H,M)\to H^q(G,M)$
    \end{itemize}
\end{definition}

\begin{proposition}
    Sia $H<G$ di indice finito. Allora $\cores^q\circ\res^q=[G:H]\id$.
\end{proposition}
\begin{corollary}
    Se $G$ è finito, vale $\#G\cdot H^q(G,M)=0$ per $q>0$.
\end{corollary}


\section{Gruppo di Brauer}
Consideriamo ora anelli $A$ con unità, non necessariamente commutativi.

\subsection{Algebre centrali semplici}

\begin{definition}
    Sia $M$ un $A$-modulo. Si dice \emph{semplice} se $M\neq0$ e non ha sottomoduli propri. $M$ si dice \emph{semisemplice} se $M=\sum_{S\subset M \text{ semplice}} S$.
\end{definition}


\begin{theorem}
    Sia $A$ anello con $1$, $M$ un $A$-modulo. Sono equivalenti
    \begin{enumerate}
        \item $\exists S_i\subset M$ semplici tali che $M=\bigoplus S_i$.
        \item $M$ è semisemplice.
        \item Per ogni $N\subset M$ esiste un $P\subset M$ tale che $M=N\oplus P$.
    \end{enumerate}
\end{theorem}

\begin{oss}
    Sottomoduli e quozienti di semisemplici sono semisemplici.
\end{oss}

\begin{definition}
    Un anello $A$ si dice semisemplice se lo è come $A$-modulo sinistro.
\end{definition}

\begin{proposition}
    Sia $A$ un anello. Sono equivalenti
    \begin{enumerate}
        \item $A$ è semisemplice.
        \item Ogni $A$-modulo è semisemplice.
        \item Ogni $A$-modulo è proiettivo.
    \end{enumerate}
\end{proposition}

\begin{lemma}[Schur]
    Siano $S,T$ moduli semplici. Allora
    \begin{itemize}
        \item Sia $\varphi:S\to T$. Si ha che $\varphi=0$ oppure è un isomorfismo.
        \item $\End(S)$ è un corpo.
    \end{itemize}
\end{lemma}

\begin{theorem}[Wedderburn]
    Sia $A$ un anello semisemplice. Allora $A=\bigoplus_i \Mat_{n_i\times n_i}(D_i)$ con $D_i$ corpi univocamente determinati.
\end{theorem}

\begin{proposition}
    Sia $E=\overline{E}$, e $D\supset E$ un corpo di dimensione finita con $E\subset Z(D)$. Allora $D=E$.
\end{proposition}

\begin{definition}
    Una $E$-algebra $A$ di dimensione finita si dice \emph{centrale} se $Z(A)=E$; si dice \emph{semplice} se $A$ non contiene ideali bilateri non banali.
\end{definition}

\begin{proposition}
    Sia $A$ una $E$-algebra di dimensione finita. Sono equivalenti
    \begin{enumerate}
        \item $A$ è semplice.
        \item $A=\Mat_{n\times n}(D)$ con $D$ corpo tale che $E\subset Z(D)$.
    \end{enumerate}
\end{proposition}

\begin{corollary}
    $A$ è un'$E$-algebra centrale semplice se e solo se $A=\Mat_{n\times n}(D)$, con $Z(D)=E$.
\end{corollary}


\begin{lemma}
    Sia $V$ un $E$-spazio vettoriale, $D$ un corpo $E$-centrale, $V_D=V\otimes_E D$ e $W\subset V_D$ un sottospazio vettoriale stabile per $D$ a destra e a sinistra. Detto $W'=W\cap V$, vale $W=W'\otimes_ED$.
\end{lemma}

\begin{theorem}
    Sia $A$ una $E$-acs; $E\subset F$ estensione di campi. Allora $F\otimes_E A$ è una $F$-acs.
\end{theorem}

\begin{theorem}
    Siano $A,A'$ delle $E$-acs. Allora anche $A\otimes_E A'$ è una $E$-acs.
\end{theorem}

\begin{definition}
    Sia $A$ una $E$-acs. Diciamo che \emph{spezza} su $F$ se $A\otimes_E F=\Mat_{n\times n}(F)$
\end{definition}

\begin{theorem}
    Sia $A$ una $E$-acs. Allora esiste un'estensione di campi $E\subset F$ finita e separabile tale che $A$ spezza su $F$.
\end{theorem}


\begin{definition}
    Sia $E$ un campo; definiamo il suo gruppo di Brauer come $\mathcal A=\faktor{ \{ E\text{-acs} \} }{\sim}$, dove $A\sim A'$ se sono algebre di matrici sullo stesso corpo.
    
    Questo è un gruppo, con l'operazione $[A]\cdot [A']=[A\otimes A']$ ed elemento neutro $[E]$.
\end{definition}

\begin{oss}
    L'elemento inverso è l'algebra opposta, in quanto $A\otimes A^{op}=\End_E(A)=\Mat_{n\times n}(E)$.
\end{oss}

\subsection{Descrizione coomologica}

\begin{lemma}
    Dato un campo $F$ vale $\Aut_F(\Mat_{n\times n}(F))=PGL_n(F)$
\end{lemma}

\begin{definition}
    Sia $E\subset L$ finita di Galois con gruppo $\Gamma$. Definiamo i seguenti oggetti:
    \begin{itemize}
        \item $\mathcal A_L=\{ [A]\in\mathcal A \st A\text{ spezza su } L \}$
        \item $\mathcal A_n=\{ A\; E\text{-acs} \st \dim_E A=n^2 \}/\text{isom.}$
        \item $\mathcal A_{n,L}= \{ A\; E\text{-acs} \st \dim_E A=n^2, A\text{ spezza su } L  \}/\text{isom.}$
    \end{itemize}
\end{definition}

\begin{oss}
    Le algebre $\mathcal A_{n,L}$ sono proprio le $E$-strutture di $\Mat_{n\times n}(L)$, pertanto c'è la corrispondenza biunivoca $\mathcal A_{n,L}\leftrightarrow H^1(\Gamma, PGL_n(L))$.
\end{oss}

\begin{definition}
    Definiamo la mappa $\delta_{n,L}:\mathcal A_{n,L}\to H^2(\Gamma,L^\ast)$ tramite la successione lunga data da $1\to L^\ast\to GL_n(L)\to PGL_n(L)\to1$.
\end{definition}

\begin{lemma}
    Siano $A\in\mathcal A_{n,L}$ e $A'\in\mathcal A_{m,L}$. Allora
    \begin{itemize}
        \item $\delta_{nm}(A\otimes A')=\delta_n(A)\cdot \delta_m(A')$.
        \item Se $[A]=[A']$ nel gruppo di Brauer, allora $\delta_n(A)=\delta_m(A')$.
        \item La mappa $\delta_k$ è surgettiva per $k=[L:E]$.
    \end{itemize}
\end{lemma}

\begin{theorem}
    Le mappe $\delta_n$ permettono di costruire l'isomorfismo di gruppi $$\mathcal A_L \cong H^2(\Gamma,L^\ast)$$
\end{theorem}


\begin{corollary}$ $
    \begin{itemize}
        \item Se $D$ è un corpo finito, allora è un campo.
        \item Il gruppo di Brauer di $\R$ è $\Z/(2)$, ovvero $\{ \R,\Hbb \}$
    \end{itemize}
\end{corollary}


\subsection{Galois infinito}

\begin{definition}
    In generale, sia $G$ un gruppo,  $H\normal G$ e $M$ un $G$-modulo. La mappa naturale $Inf^q:H^q\left(\faktor G H, M^H\right)\rightarrow H^q(G,M)$ è detta \emph{inflazione}.
\end{definition}

Consideriamo ora $E\subset F$ di Galois con gruppo $\Gamma$; per ogni $E\subset L$ finita di Galois indichiamo $\Sigma_L=\Gal{F}{L}$, $\Gamma_L=\Gal{L}{E}=\faktor{\Gamma}{\Sigma_L}$.

\begin{definition}
    Un $\Gamma$-modulo $M$ è detto liscio se vale $M=\bigcup M^{\Sigma_L}$, dove l'unione è fatta sulle $L$ finite di Galois.
\end{definition}

\begin{definition}
    Possiamo definire $H^q_{cont}(\Gamma, M)=\varinjlim H^q(\Gamma_L,M^{\Sigma_L})$, dove il sistema diretto è dato dalle $L$ estensioni finite di Galois e le mappe sono le inflazioni.
\end{definition}

\begin{lemma}
    Sia $0\to A_i\to B_i\to C_i\to 0$ una successione esatta per ogni indice $i$; allora vale $0\to \varinjlim A_i\to \varinjlim B_i\to \varinjlim C_i\to0$.
\end{lemma}

\begin{lemma}
    Sia $M$ un $\Gamma$-modulo continuo, e $M=\bigcup M_\alpha$ con $\alpha$ insieme filtrante. Allora $\varinjlim H^q(\Gamma, M_\alpha)=H^q_{cont}(\Gamma, M)$.
\end{lemma}

\begin{theorem}
    Sia $0\to A\to B\to C\to0$ una successione esatta di moduli lisci. Allora ho la successione esatta lunga in coomologia con gli $H^q_{cont}$.
\end{theorem}

\begin{theorem}
    Sia $E$ un campo, $F$ la sua chiusura separabile e $\Gamma=\Gal{F}{E}$. Allora il gruppo di Brauer di $E$ è $\mathcal A=\mathcal A_F\cong H^2(\Gamma, F^\ast)$.    
\end{theorem}

\section{Costruzione dei funtori derivati}

\subsection{Risoluzione iniettiva}

\begin{definition}
    Un oggetto $I$ è detto iniettivo se $$\begin{tikzcd}
    0 \ar[r] & X \ar[d] \ar[r] & Y \ar[dl, dashed] \\
    & I
    \end{tikzcd}$$
\end{definition}
\begin{oss}
    Prodotto di iniettivi è iniettivo.
\end{oss}

\begin{definition}
    Un $A$-modulo $I$ è \emph{divisibile} se $\forall x\in I,a\in A\setminus 0$ esiste un $y\in I$ tale che $ay=x$. 
\end{definition}

\begin{theorem}
    Se $A$ è un PID e $I$ è un $A$-modulo, allora $I$ è iniettivo se e solo se è divisibile
\end{theorem}
\begin{corollary}
    $\Q$ e $\Q/\Z$ sono degli $\Z$-moduli iniettivi.
\end{corollary}

\begin{proposition}
    Sia $M$ uno $\Z$-modulo. Allora esiste un modulo iniettivo $I$ e un'immersione $\varphi:M\to I$.
\end{proposition}

\begin{lemma}
    Sia $M$ un $A$-modulo e $N$ uno $\Z$-modulo. Allora vale $\Hom_\Z(M,N)\cong\Hom_A(M,N_A)$, dove $N_A=\Hom_\Z(A,N)$ che è un $A$-modulo con l'azione $(a\cdot\varphi)(x)=\varphi(xa)$.
    
    In particolare il funtore $N_A$ manda iniettivi in iniettivi.
\end{lemma}

\begin{theorem}
    Sia $M$ un $A$-modulo. Allora esiste un $A$-modulo iniettivo $I$ tale che $0\to M\to I$. Ovvero $Mod_A$ ha abbastanza iniettivi.
\end{theorem}

\begin{corollary}
    Ogni $A$-modulo ammette una risoluzione iniettiva, costruita nel seguente modo
    $$\begin{tikzcd}
    & & & & & \faktor{I^1}{I^0} \ar[dr] & \\
    0 \ar[r] & M \ar[r] & I^0 \ar[dr] \ar[rr]  & & I^1 \ar[ur] \ar[rr] & & I^2\\
    & & & \faktor{I^0}{M} \ar[ur] & & &
    \end{tikzcd}$$
\end{corollary}


\begin{theorem}
    Sia $\mathcal C$ la categoria degli $A$-moduli. Sia $\mathcal F\subset \mathcal C$ una classe di moduli tali che $\forall M\in\mathcal C\;\exists I\in\mathcal F$ tale che $0\to M\to I$.
    
    Allora dato un $M^\bullet\in \Com^+(\mathcal C)$ esiste $X^\bullet\in\Com^+(\mathcal F)$ con una mappa $\varphi:M^\bullet\to X^\bullet$ che è un quasi isomorfismo e iniettiva in ogni grado.
\end{theorem}


\subsection{Categorie triangolate}

\begin{definition}
    Sia $\mathcal C$ una categoria additiva con un funtore invertibile $[1]$, e una famiglia di triangoli distinti $X\to Y\to Z\to X[1]$. $\mathcal C$ si dice \emph{pretriangolata} se valgono
    
    \begin{itemize}
        \item[TR1]\begin{enumerate}[a)]
            \item $\forall X\in\mathcal C$ il triangolo $X=X\to0\to X[1]$ è distinto.
            \item $\forall \varphi:X\to Y$ esiste un triangolo distinto $X\to Y\to Z\to X[1]$.
            \item Un triangolo isomorfo ad un distinto è ancora distinto.
        \end{enumerate}
        \item[TR2] Il triangolo $X\to Y\to Z\to X[1]$ è distinto se e solo se $Y\to Z\to X[1]\to Y[1]$ è distinto.
        \item[TR3] $$\begin{tikzcd}
        X \ar[r] \ar[d] & Y \ar[r] \ar[d] & Z \ar[r] \ar[d, dashed] & X[1] \ar[d]\\
        X' \ar[r] & Y' \ar[r] & Z' \ar[r] & X'[1]\\
        \end{tikzcd}$$ 
    \end{itemize}
\end{definition}

\begin{theorem}
    Se $\mathcal C$ è abeliana, la categoria $\Kom^+(\mathcal C)$, con funtore $(X[1])^n=X^{n+1}$ e $\de_{X[1]}^n=-\de_X^{n+1}$, e triangoli distinti della forma $X\xrightarrow{\varphi} Y\to C(\varphi)\to X[1]$ dati da $C(\varphi)^n=Y^{n}\oplus X^{n+1}$ e $\de_{C(\varphi)}^n=\begin{pmatrix}
    \de_Y & \varphi \\ 0 & -\de_X
    \end{pmatrix}$ è pretriangolata.
\end{theorem}

\begin{proposition}
    Sia $X\to Y\to Z\to X[1]$ un triangolo distinto in una categoria pretriangolata. Allora $\forall U$ la successione $\Hom(U,X)\to \Hom(U,Y)\to \Hom(U,Z)\to \Hom(U,X[1])$ è esatta di gruppi abeliani.
\end{proposition}

\begin{proposition}
    Sia $X^\bullet\to Y^\bullet\to Z^\bullet\to X^\bullet[1]$ un triangolo distinto nella categoria omotopica degli $A$-moduli. Allora per ogni $i$ la successione $H^i(X)\to H^i(Y)\to H^i(Z)\to H^{i+1}(X)$ è esatta.
\end{proposition}

\begin{proposition}
    Sia $0\to X^\bullet\xrightarrow{\varphi} Y^\bullet\xrightarrow{\pi} Z^\bullet\to 0$ una successione esatta di complessi. Consideriamo la mappa $F:C(\varphi)\to Z$ data da $F(y,x)=\pi(y)$. Questa è un quasi isomorfismo, e inoltre fa commutare il diagramma
    $$\begin{tikzcd}
    H^i(X) \ar[r] \ar[d, equal] & H^i(Y) \ar[r] \ar[d, equal]& H^i(Z) \ar[r, "-\omega"] & H^{i+1}(X) \ar[d, equal]\\
    H^i(X) \ar[r] & H^i(Y) \ar[r] & H^i(C(\varphi)) \ar[u, "H^i(F)"] \ar[r] & H^{i+1}(X)&
    \end{tikzcd}$$
\end{proposition}

\subsection{Funtori derivati}
\begin{lemma}
    Sia $X^\bullet$ un complesso esatto, e $I^\bullet$ un complesso di oggetti iniettivi. Allora $\Hom_\Kom(X,I)=0$.
\end{lemma}

\begin{proposition}
    Siano $A,B$ complessi e $I$ complesso iniettivo.
    \begin{enumerate}
        \item Se $\varphi:A\to B$ è un quasi isomorfismo, allora $\Hom_\Kom(B,I)\to\Hom_\Kom(A,I)$ è un isomorfismo.
        \item La risoluzione iniettiva di un complesso è unica in $\Kom$ a meno di unico isomorfismo.
        \item Dette $A\to I_A$ e $B\to I_B$ le risoluzioni iniettive, c'è una mappa iniettiva $\Hom_\Kom(A,B)\hookrightarrow \Hom_\Kom(I_a,I_B)$. 
    \end{enumerate}
\end{proposition}


\begin{definition}
    Sia $F:\mathcal A\to \mathcal B$ un funtore additivo di categorie abeliane; supponiamo che $\mathcal A$ abbia abbastanza iniettivi.
    
    Per ogni $X\in\Com^+(\mathcal A)$ esiste una (unica in $\Kom$) risoluzione iniettiva $I_X$.
    
    Definiamo allora $RF:\Kom^+(\mathcal A)\to\Kom^+(\mathcal B)$ tramite $RF(X)=F(I_X)$.
    
    Sia poi $R^iF:\Kom^+(\mathcal A)\to\mathcal B$ dato da $R^iF(X)=H^i(RF(X))$.
\end{definition}

\begin{oss}
    Se $F$ è esatto a sinistra, allora $R^0F(X)=F(X)$.
\end{oss}

\begin{proposition}
    Il funtore $RF$ manda triangoli distinti in triangoli distinti.
\end{proposition}


\begin{definition}
    Sia $\mathcal A$ categoria abeliana, $X\in\mathcal A$. Definiamo $F_X(Y)=\Hom(X,Y)$ e $\underline{F_X}(Y)=\Hom(Y,X)$.
    
    Definiamo allora
    \begin{itemize}
        \item $\Ext^i(X,Y)=R^iF_X(Y)$ per $Y\in\Com^+(\mathcal A)$.
        \item $\underline{\Ext^i}(X,Y)=R^i\underline{F_Y}(X)$ per $X\in\Com^+(\mathcal A)$ (devo usare una risoluzione proiettiva, perché è controvariante).
    \end{itemize}
\end{definition}


\begin{definition}
    Un complesso doppio è un insieme di oggetti $X^{i,j}$ e mappe $\de_O^{i,j}$ e $\de_V^{i,j}$ messi così:
    $$\begin{tikzcd}
    & \, & \, &  \\
    \ar[r] &  X^{i,j+1} \ar[u] \ar[r] & X^{i+1,j+1} \ar[r] \ar[u] &  \, \\
    \ar[r] &  X^{i,j} \ar[u, "\de_V^{i,j}"] \ar[r, "\de_O^{i,j}"] & X^{i+1,j} \ar[r] \ar[u] &  \,  \\
    & \ar[u] & \ar[u] & 
    \end{tikzcd}$$
    
    Sia poi $T^n=\bigoplus_{i+j=n}X^{i,j}$ il complesso totale, con bordi $\rest{\de_T^n}{X^{i,j}}=\de_O^{i,j}+(-1)^i\de_V^{i,j}$.
\end{definition}

\begin{proposition}
    Sia $X^{i,j}$ un complesso doppio con $X^{i,j}=0$ per $i<0$ o $j<0$. Supponiamo inoltre che le rige e le colonne siano esatte, tranne al più in $0$; definiamo $A^j=\ker\de_O^{0,j}$ e $B^i=\ker\de_V^{i,0}$.
    Siano poi $\alpha:A^\bullet\to T$ e $\beta:B^\bullet\to T$ le inclusioni.
    
    Allora $\alpha,\beta$ sono mappe di complessi e quasi isomorfismi.
\end{proposition}

\begin{proposition}
    Siano $X,Y$ oggetti. Allora $\Ext^i(X,Y)=\underline{\Ext^i}(X,Y)$
\end{proposition}



