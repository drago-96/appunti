\documentclass[a4paper]{book}

\usepackage[utf8]{inputenc}
\usepackage[italian]{babel}
\usepackage{amsmath}
\usepackage{amssymb}
\usepackage{amsthm}
\usepackage{commath}
\usepackage{mathtools}
\usepackage{faktor}
\usepackage[]{algorithm2e}
\usepackage{hyperref}
\usepackage{color}
\usepackage{graphicx}
\usepackage{multirow}
\usepackage{mathrsfs}
\usepackage{array}
\usepackage{rotating}
\usepackage{multirow}
\usepackage{breqn}
\usepackage{tikz-cd}
%\setcounter{secnumdepth}{5}
\setlength\extrarowheight{3pt}

\usepackage{hyperref}

\hypersetup{
    colorlinks,
    citecolor=black,
    %filecolor=black,
    linkcolor=black,
    urlcolor=black
}



\newtheorem{theorem}{Teorema}[section]
\newtheorem*{theorem*}{Teorema}
\newtheorem{acknowledgement}[theorem]{Acknowledgement}
\newtheorem{congettura}[theorem]{Congettura}
%\newtheorem{algorithm}[theorem]{Algoritmo}
\newtheorem{axiom}[theorem]{Assioma}
\newtheorem{case}[theorem]{Caso}
\newtheorem{claim}[theorem]{Claim}
\newtheorem{conclusion}[theorem]{Conclusione}
%\newtheorem{condition}[theorem]{Condizione}
\newtheorem{conjecture}[theorem]{Congettura}
\newtheorem{corollary}[theorem]{Corollario}
\newtheorem{criterion}[theorem]{Criterio}
\newtheorem{lemma}[theorem]{Lemma}
\newtheorem{notation}[theorem]{Notazione}
\newtheorem{problem}[theorem]{Problema}
\newtheorem{proposition}[theorem]{Proposizione}
\newtheorem{summary}[theorem]{Riassunto}

\theoremstyle{definition}
\newtheorem{definition}[theorem]{Definizione}
\newtheorem{example}[theorem]{Esempio}
\newtheorem{exercise}{Esercizio}[section]
\newtheorem{solution}[exercise]{Soluzione}
\newtheorem*{notazione}{Notazione}


\newcommand{\todo}[1]{ \marginpar{\textbf{TODO:} \textcolor{red}{#1}} }

\theoremstyle{remark}
%\newtheorem*{remark}[theorem]{Osservazione}
\newtheorem*{oss}{Osservazione}

\addtolength{\topmargin}{-.5in}
\addtolength{\textheight}{0.5in}





\newenvironment{mysol}{%
    \begin{sol}$ $\nobreak\ignorespaces
    }{%
        \qedhere
    \end{sol}
}

\DeclareMathOperator{\lcm}{lcm}
\DeclareMathOperator{\ann}{Ann}
\DeclareMathOperator{\lt}{lt}
\DeclareMathOperator{\Lt}{Lt}
\DeclareMathOperator{\Deg}{Deg}
\DeclareMathOperator{\Syl}{Syl}
\DeclareMathOperator{\Ris}{Ris}
\DeclareMathOperator{\Imm}{Im}
\DeclareMathOperator{\Dom}{Dom}
\DeclareMathOperator{\car}{char}
\DeclareMathOperator{\p}{\mathcal P}
\DeclareMathOperator{\st}{\; |\;}
\DeclareMathOperator{\et}{\;\wedge\;}
\DeclareMathOperator{\tr}{Tr}
\DeclareMathOperator{\n}{N}
\DeclareMathOperator{\disc}{disc}
\DeclareMathOperator{\GL}{GL}
\DeclareMathOperator{\SL}{SL}
\DeclareMathOperator{\Spec}{Spec}
\DeclareMathOperator{\Hom}{Hom}
\DeclareMathOperator{\End}{End}
\DeclareMathOperator{\Aut}{Aut}
\DeclareMathOperator{\h}{ht}
\DeclareMathOperator{\coh}{coht}
\DeclareMathOperator{\id}{id}
\DeclareMathOperator{\stab}{stab}
\DeclareMathOperator{\coker}{coker}
\DeclareMathOperator{\Com}{Com}
\DeclareMathOperator{\Kom}{Kom}
\DeclareMathOperator{\Ext}{Ext}
\DeclareMathOperator{\Tor}{Tor}
\DeclareMathOperator{\coind}{coInd}
\DeclareMathOperator{\ind}{Ind}
\DeclareMathOperator{\cores}{coRes}
\DeclareMathOperator{\res}{Res}
\DeclareMathOperator{\Mat}{Mat}


\newcommand{\m}{\mathfrak{m}}
\newcommand{\N}{\mathbb{N}}
\newcommand{\Z}{\mathbb{Z}}
\newcommand{\Q}{\mathbb{Q}}
\newcommand{\R}{\mathbb{R}}
\newcommand{\C}{\mathbb{C}}
\newcommand{\D}{\mathcal{D}}
\newcommand{\Hbb}{\mathbb{H}}
\newcommand{\A}{\mathbb{A}}
\newcommand{\F}{\mathbb{F}}
\newcommand{\Zn}[1]{\Z/#1\Z}
\newcommand{\gen}[1]{\ensuremath{\left< #1\right>}}
\newcommand{\normal}{\mathrel{\unlhd}}
\newcommand{\sing}[1]{\{#1\}}
\newcommand{\ds}{\displaystyle}
\newcommand{\eps}{\varepsilon}
\newcommand{\de}{\partial}

\renewcommand{\norm}[1]{\left\lVert#1\right\rVert}
\newcommand{\vp}[1]{\upsilon_p\left(#1\right)}
\newcommand{\np}[1]{\left|#1\right|_p}
\newcommand{\pf}{\mathfrak{p}}
\newcommand{\qf}{\mathfrak{q}}
\newcommand{\Pf}{\mathfrak{P}}
\newcommand{\Oc}{\mathcal{O}}
\newcommand{\vpf}[1]{\upsilon_\pf\left(#1\right)}
\newcommand{\vPf}[1]{\upsilon_\Pf\left(#1\right)}
\newcommand{\npf}[1]{\left|#1\right|_\pf}
\newcommand{\rest}[2]{\left. #1 \right|_{#2}}

\newcommand{\Gal}[2]{\operatorname{Gal}\left(\faktor{#1}{#2}\right)}
%\nolinks

\DeclareMathOperator{\wt}{wt}

\author{Riccardo Zanotto}
\title{Crittografia}

\begin{document}
    \maketitle

    \tableofcontents

    \chapter{NT algs}
    \section{Conti di base}
    Alcuni algoritmi standard e trick vari per velocizzare i conti.

    \subsection{Alg euclideo esteso}
    Dati $a,b\in A$ anello che possiede una divisione euclidea, posso trovare $u,v$ tali che $ua+vb=\gcd(a,b)$.

    Definisco $v_0=\begin{pmatrix}a \\ 1 \\ 0\end{pmatrix}, v_1=\begin{pmatrix}b \\ 0 \\ 1\end{pmatrix}$ e poi una successione per ricorrenza $v_{i+1}=v_{i-1}-q_iv_i$ dove $r_{i-1}=q_ir_i+r_{i+1}$ è la divisione con resto e $r_i$ è la prima coordinata di $v_i$.

    Detti $s_i,t_i$ la seconda e la terza componente di $v_i$, valgono un po' di cose:
    \begin{itemize}
        \item $r_i=as_i+bt_i$
        \item Eventualmente $r_{k+1}=0$ e allora $r_k=\gcd(a,b)$
        \item Ad ogni passo $s_i,t_i$ sono ``piccoli"
    \end{itemize}

    \textbf{Costo computazionale}: Supposto $a>b$, il numero di iterazioni è $O(\log a)$.\\
    Il costo totale dunque può essere stimato con $O(\log^3 a)$.

    \begin{oss}
        Questo vuol dire che trovare l'inverso modulo $n$ costa $\log^3n$
    \end{oss}

    \subsection{Fast mod exp}
    Vogliamo calcolare $b^n \pmod m$. Scriviamo $n$ in base $2$ e calcoliamo con quadrati ripetuti le potenze $b,b^2,b^4,b^8,\dots$ (sempre riducendo modulo $m$), moltiplicando quando ci sono gli $1$ in $n$.

    \textbf{Costo computazionale}: $O(\log n\cdot \log^2m)$. Stiamo facendo $\log n$ moltiplicazioni tra numeri grossi al più $m^2$.

    \subsection{Montgomery multiplication}
    Per calcolare $ab\pmod m$ si fa il conto negli interi e poi la divisione con resto per $m$, che è un po' lenta.

    Scegliamo allora un $r>m$ del tipo $r=10^k$ (o comunque per cui è facile ridurre un numero).

    Precalcoliamo $rr'=1+mm'$ con $0<r'<m$ e $0<m'<r$.

    \begin{lemma}
        Dato un $x<mr$ so calcolare $xr'\pmod m$ solo con divisioni per $r$ che sono veloci.
    \end{lemma}
    Sia $s=xm'\mod r$. Allora $sm\equiv xmm'\pmod{mr}$, e aggiungendo $x$ si ha $x+sm\equiv x(1+mm')\equiv xrr'\pmod{rm}$.\\
    Perciò vale $z=\frac{x+sm}{r}\in\Z$ ed è proprio quello che cercavamo $z\equiv xr'\pmod m$.

    \bigskip
    A questo punto se devo fare tanti conti con $a_1,\dots,a_n$ modulo $m$, calcolo subito $w=r^2\mod m$ (con la divisione solita); poi porto tutto in rappresentanti di Montgomery: $b_i\equiv a_ir\pmod m$ e questo lo faccio tramite $a_ir\equiv a_iwr'\pmod m$.

    Il rappresentante di una somma è banalmente la somma; il rappresentante del prodotto è facile da calcolare: $xyr\equiv(xr)(yr)r'\pmod m$. Quindi se ho i rappresentanti di $x,y$ grazie al lemma posso trovare in fretta il rappresentante di $xy$.

    Finiti tutti i conti posso di nuovo ritrasformare il risultato nella sua forma standard.

    \subsection{Quadrati mod $p$}

    \textbf{Simboli di Legendre/reciprocità} Sono noti. Lo abbiamo fatto con il conto su $G=\sum_{i=0}^{p-1}\left( \frac{i}{p} \right)\xi^i$.

    \bigskip
    Vogliamo anche trovare le radici quadrate modulo $p$. Cioè risolvere $x^2\equiv a\pmod p$ con $a$ residuo quadratico.

    \bigskip
    \textbf{Algoritmo di Cipolla}: Prendiamo un $n$ tale che $n^2-a$ non sia un quadrato.\\
    Sia $w=\sqrt{n^2-a}$ nel campo $\F_{p^2}$. Allora $z=(n+w)^{\frac{p+1}{2}}\in\F_p$ è una radice quadrata di $a$.

    \textbf{Complessità}: $O(\log^3p)$.


    \bigskip
    \textbf{Algoritmo di Tonelli-Shanks}: Prendiamo $n$ un nonresiduo. Scriviamo $p-1=2^\alpha\cdot s$; calcoliamo $b=n^s\mod p$ e $r=a^{\frac{s+1}{2}} \mod p$.\\
    Osserviamo che $b$ è una radice $2^\alpha$-esima primitiva dell'unità, poiché non è un quadrato.\\
    Inoltre $(r^2a^{-1})^{2^{\alpha-1}}\equiv (a^s)^{2^{\alpha-1}}\equiv a^{\frac{p-1}{2}}\equiv1\pmod p$, perciò $r^2a^{-1}\equiv b^{-2j}\pmod p$ per qualche $j<2^{\alpha-1}$, cioè $a\equiv(rb^j)^2\pmod p$.\\
    Vogliamo allora trovare $j$ e lo facciamo per induzione, trovandone le cifre binarie $j=j_0+2j_1+\dots+j_{\alpha-2}2^{\alpha-2}$, in modo che $(b^{j_0+\dots+j_{k-1}2^{k-1}}r)^2a^{-1}$ sia una radice $2^{\alpha-k-1}$-esima.\\
    Vale $\left(  (b^{j_0+\dots+j_{k-1}2^{k-1}}r)^2a^{-1} \right)^{2^{\alpha-k-2}}\equiv b^{-j_k2^{\alpha-1}}\equiv(-1)^{j_k}\pmod p$, dunque calcolando LHS trovo $j_k=0,1$ a seconda se la potenza fa $1,-1$.

    \textbf{Complessità}: $O(\log^2 p(\log p+\alpha^2))$ se conosco già $n$.

    \subsection{Karatsuba}
    Finora abbiamo detto che per moltiplicare due interi $x,y$ di $n$ cifre servono $n^2$ operazioni.

    Tuttavia si può fare meglio: fissati un qualche $B,m$ possiamo scrivere $x=x_1B^m+x_0,y=y_1B^m+y_0$ e osservare che $xy=x_1y_1B^{2m}+B^m(x_0y_1+x_1y_0)+x_0y_0$.

    Per calcolare questo numero servono $4$ moltiplicazioni (e degli shift in base $B$), ma è possibile farlo in 3 poiché $x_0y_1+x_1y_0=(x_0+x_1)(y_0+y_1)-x_0y_0-x_1-y_1$.

    \textbf{Complessità}: Alla fine arriviamo a $O(n^{\log_2 3})$, scegliendo $B=2$ e per ricorsione $m=n/2$.

    \section{Test di primalità}
    PRIMES in P. Ma non l'abbiamo fatto :(

    \subsection{Pseudoprimi}

    Uno dei test più ovvi per verificare se $n$ è primo e vedere se $b^{n-1}\equiv 1\pmod n$; se la congruenza fallisce, $n$ è composto, altrimenti diciamo che $n$ è uno \emph{pseudoprimo} rispetto a $b$.

    \begin{proposition}
        Se $n$ è uno pseudoprimo rispetto ad almeno un $b$, allora è uno pseudoprimo per almeno metà dei $b\in(\Z/n\Z)^\ast$.
    \end{proposition}

    \begin{definition}
        Un intero $n$ si dice di \emph{Carmichael} se è uno pseudoprimo rispetto a ogni $b\in(\Z/n\Z)^\ast$.
    \end{definition}

    Chi sono i numeri di Carmichael?
    \begin{proposition}
        Sia $n$ un intero dispari. Se $n$ non è squarefree, non è di Carmichael.\\
        Se $n$ è squarefree, $n$ è di Carmichael se e solo se $p-1\mid n-1\;\forall p\mid n$
    \end{proposition}

    Ecco un altro tipo di pseudoprimi:
    \begin{definition}
        Diciamo che $n$ intero dispari è un \emph{pseudoprimo di Eulero} rispetto a $b$ se vale $$\left( \frac b n \right)\equiv b^{\frac{n-1}{2}}\pmod n$$
    \end{definition}

    \begin{proposition}
        Ogni $n$ è pseudoprimo di Eulero per al più metà dei $b$ coprimi con $n$.
    \end{proposition}

    Abbiamo allora l'algoritmo di \textbf{Solovay-Strassen}: scegliamo $k$ interi $0<b<n$ random. Se $n$ non è uno pseudoprimo rispetto a un qualche $b$, allora $n$ è composto. Se invece $n$ è pseudoprimo per tutti, allora è primo con probabilità circa $1-2^{-k}$.

    \subsection{Miller-Rabin}
    \begin{definition}
        Sia $n$ un intero dispari, e scriviamo $n-1=2^st$ con $t$ dispari. Diciamo allora che $n$ è un \emph{pseudoprimo forte} rispetto a $b$ se $b^t\equiv\pmod n$, oppure $b^{t2^r}\equiv-1\pmod n$ per qualche $r<s$.
    \end{definition}

    \begin{proposition}
        Se $n$ è primo, è uno pseudoprimo forte per tutti i $b$.\\
        Se $n$ è composto, allora è pseudoprimo forte per al più $1/4$ dei possibili $b$.
    \end{proposition}

    Il test consiste dunque dei seguenti step:
    \begin{enumerate}
        \item Scrivo $n-1=2^st$.
        \item Scelgo un $b$ random; calcolo $a\equiv b^t\pmod n$. Se $a\equiv1\pmod n$, restituisco ``forse primo''.
        \item Faccio $a\mapsto a^2\pmod n$ finché non trovo un $-1$ restituendo ``forse primo''.
        \item Se non ho trovato nessun $-1$ restituisco ``composto''.
        \item Se ho ottenuto ``forse primo'' rifaccio dal punto 2.
    \end{enumerate}

    A questo punto se ho eseguito il punto 2 almeno $k$ volte e non ho mai ottenuto ``composto'', so che $n$ è primo con probabilità cir a $1-4^{-k}$.

    \subsection{Lucas e Pocklington-Lehmer}

    \begin{proposition}
        Fissato un $n$ intero positivo. Supponiamo che esista un $1<a<n$ tale che $a^{n-1}\equiv1\pmod n$ e inoltre $a^{\frac{n-1}{q}}\not\equiv1\pmod n$ per ogni $q\mid n-1$ fattore primo. Allora $n$ è primo
    \end{proposition}
    La seconda condizione infatti dice che $(\Z/n\Z)^\ast$ ha ordine $n-1$.

    Tuttavia occorre fattorizzare $n-1$, che può essere difficile a piacere.

    \begin{proposition}
        Sia $n$ un intero, e supponiamo che esistano $a$ e $p$ primo tali che:
        \begin{itemize}
            \item $a^{n-1}\equiv1\pmod n$
            \item $p\mid n-1$ e $p>\sqrt n -1$
            \item $\gcd(a^{\frac{n-1}{p}}-1,n)=1$
        \end{itemize}
        Allora $n$ è primo.
    \end{proposition}
    Anche queste condizioni sono difficili da soddisfare in realtà, perché potrebbe non esistere un $p$ che soddisfa la seconda condizione.

    \begin{proposition}
        Sia $n$ un intero, e scriviamo $n-1=ab$ con $a>\sqrt n$ e di cui conosciamo la fattorizzazione. Supponiamo che per ogni $p\mid a$ primo esiste un $m_p$ tale che $m_p^{n-1}\equiv1\pmod n$ e $\gcd(m_p^{\frac{n-1}{p}}-1,n)=1$. Allora $n$ è primo.
    \end{proposition}

    \bigskip
    Dunque alla fine il test di Pocklington consiste nel trovare molti fattori piccoli di $n-1$ sperando di superare $\sqrt n$ (e questa è la parte molto difficile). A quel punto si cercano degli $a_p$ che soddisfino le condizioni; spesso $a_p=2$ basta già da solo.


    \section{Fattorizzazione di polinomi}
    Per i polinomi ci pensa \href{http://people.dm.unipi.it/gianni/TC&C/fattorizzazione.pdf}{Knuth}.

    \subsection{Polinomi su $\F_q$}
    \subsubsection{Algoritmo di Berlekamp}
    Abbiamo $f\in\F_q[x]$ di grado $n$; possiamo supporlo squarefree dividendo per $\gcd(f,f')$, ovvero $f=f_1\cdots f_r$.

    Consideriamo la mappa $\varphi:\faktor{\F_q[x]}{f(x)}\to \faktor{\F_q[x]}{f(x)}$ data dall'elevamento alla $q$. Per il teorema cinese, la mappa $\varphi-\id$ è un endomorfismo di $\faktor{\F_q[x]}{f_1(x)}\times\dots\times\faktor{\F_q[x]}{f_r(x)}$ che è prodotto di campi finiti di grado potenze di $q$; in particolare $\ker(\varphi-\id)=\F_q^r$.

    D'altra parte osserviamo che $v\in\ker(\varphi-\id)$ se e solo se $v(x)^q\equiv v(x)\pmod{f(x)}$; inoltre $v^q-v=\prod_{s\in\F_q}(v-s)$. Dato che $\deg v<\deg f$, otteniamo una fattorizzazione non banale $f(x)=\prod_{s\in\F_q}\gcd(f(x),v(x)-s)$.

    Per trovare questi $v$ basta calcolare il nucleo di una matrice, in particolare quella data dal cambio base $x^{iq}\equiv Q_{i,n-1}x^{n-1}+\dots+Q_{i,0}\pmod{f(x)}$.

    L'algoritmo è allora dato da
    \begin{enumerate}
        \item Divido $f$ per $\gcd(f,f')$.
        \item Creo la matrice $Q$ di cambio base
        \item Trovo una base del nucleo di $Q-I$ con operazioni elementari
        \item Calcolo tutti i $\gcd(f,v-s)$ con $v\in\ker(\varphi-\id)$ e $s\in\F_q$.
    \end{enumerate}

    Per $q$ piccolo, allora i tempi di esecuzione sono di
    \begin{enumerate}
        \item $O(n^2)$ tramite algoritmo euclideo.
        \item $O(qn^2)$: calcolo per ricorrenza i coefficienti di $x^k$ per $k=1,\dots,qn$.
        \item $O(n^3)$ con triangolazione gaussiana.
        \item $O(qrn^2)$: provo tutti i $\gcd$ con tutti gli $s\in\F_q$.
    \end{enumerate}

    Per $q$ grosso, le moltiplicazioni costano $\log^2q$, e lo step $4$ chiede di provare troppi valori di $s$. Inoltre facciamo lo step $2$ con la fast-exp (per fare il quadrato ci basta tenere i coefficienti fino a $x^{2n}$).

    Usiamo poi il seguente step 4':

    Dato $v\in\ker(\varphi-\id)$ sappiamo $f\mid v^p-v=v(v^{\frac{p-1}{2}}-1)(v^{\frac{p-1}{2}}+1)$; abbastanza spesso calcolando $\gcd(f,v^{\frac{p-1}{2}}-1)$ otteniamo un fattore non banale.\\
    In particolare se $v(x)\equiv s_j\pmod{f_j(x)}$, vediamo che $f_j\mid v^{\frac{p-1}{2}}-1 $ se e solo se $s_j$ è un quadrato, e questo accade circa $q/2$ volte. La probabilità che scelto un $v$ a caso, il $\gcd$ scritto sopra ci dia informazioni non banali è esattamente $1-\left(\frac{q-1}{2q}\right)^r-\left(\frac{q+1}{2q}\right)^r\ge\frac49$.

    Perciò dopo $O(\log r)$ pesche casuali di $v$ abbiamo trovato tutti gli $r$ fattori di $f$; il tempo totale di questo step 4' è di $O(n^2\log^3q\log r)$.

    \subsubsection{Cantor-Zassenhaus}

    Osserviamo che possiamo calcolare facilmente una fattorizzazione di $f=F_1\cdots F_s$ con $F_i=\prod_{\deg j_f=i}f_j$, cioè una fattorizzazione di $f$ in parti con fattori dello stesso grado. Questo perché vale $\ds x^{q^d}-x=\prod_{\substack{\deg g\mid d \\ g\text{ irr}}}g(x)$, e allora ottengo $F_1=\gcd(f,x^q-x)$, e poi $F_{i+1}=\gcd(\frac{f}{F_1\cdots F_i},x^{q^{i+1}}-x)$.

    L'algoritmo di Cantor-Zassenhaus fattorizza poi ciascuno degli $F_i$, usando la formula
    $$ F_i(x) = \gcd(F_i,t)\cdot \gcd(F_i, t^{\frac{q^d-1}{2}}-1)\cdot \gcd(F_i, t^{\frac{q^d-1}{2}}+1)$$
    valida per ogni $t\in\F_q[x]$, poiché $t(\alpha)^{q^d}=t(\alpha)$ per ogni $\alpha$ di grado $d$ su $\F_q$.

    Scelto un $t(x)$ random di grado $\le 2d-1$, la formula sopra dà un fattore non banale circa il 50\% delle volte

    \subsection{Polinomi su $\Q$}
    (leggere sul \href{http://people.dm.unipi.it/gianni/TC&C/Hensel.pdf}{Childs})

    La strategia è fattorizzare $f\in\Z[x]$ modulo un $M$ molto grosso, in particolare più del doppio dei possibili valori assoluti di coefficienti di fattori di $f$. A quel punto i fattori che abbiamo trovato o corrispondono a fattori veri in $\Z[x]$, oppure $f$ è irriducibile.

    Ci servono due cose: il bound e un modo per fattorizzare modulo $M$. Potremmo prendere $M$ primo e usare Berlekamp, ma useremo un altro metodo.

    \begin{lemma}[sollevamento di Hensel]
        Sia $f\in\Z[x]$ monico tale che $f\equiv g_1h_1\pmod m$, con $g_1,h_1$ coprimi modulo $m$.\\
        Allora esistono polinomi monici $g_2,h_2\in\Z[x]$ tali che $g_1\equiv g_2\pmod m$, $h_1\equiv h_2\pmod m$ e $f\equiv g_2h_2\pmod{m^2}$; inoltre $g_2,h_2$ sono coprimi modulo $m^2$ e sono unici modulo $m^2$.
    \end{lemma}
    \begin{proof}
        Sappiamo che $f=g_1h_1+mk$ con $\deg k<\deg(g_1h_1)$. Cerchiamo $g_2=g_1+mb, h_2=h_1+mc$.\\
        Mi serve $k\equiv bh_1+cg_1\pmod m$; dato che $h_1,g_1$ sono coprimi, trovo $b,c$ di grado basso.\\
        Per vedere che sono coprimi solleviamo anche la relazione $r_1g_1+s_1h_1\equiv 1\pmod m$.
    \end{proof}

    Quindi una volta fattorizzato $f$ su $\F_p$ in pochi fattori irriducibili, posso sollevare la fattorizzazione ad ogni $p^{2^\ell}$, finché non supero il bound sui coefficienti.

    \begin{proposition}[Mignotte]
        Sia $f=\sum_{i=0}^n a_ix^i$ e $g=\sum_{i=0}^d b_ix^i$. Allora se $g\mid f$ vale $\sum_{i=0}^d |b_i|\le \left|\frac{b_d}{a_n}\right|2^d\sqrt{\sum_{j=0}^n a_j^2}$.
    \end{proposition}
    (serve che la misura di Mahler di un polinomio è $\ge$ della sua norma $L^2$, vedi \href{http://people.dm.unipi.it/gianni/TC&C/Mignotte.pdf}{qua})

    \section{Fattorizzazione in $\Z$}
    Sia $n$ il numero da fattorizzare

    \subsection{Pollard's $\rho$}
    Consideriamo una funzione pseudo-random $f:\Z/n\Z\to\Z/n\Z$, tipo $f(x)=x^2+1$.\\
    Prendiamo allora la successione $a_{i+1}=f(a_i)$ con $a_0$ scelto da noi.\\
    Se $p\mid n$ è un fattore, allora da un certo punto in poi gli $a_i$ sono periodici modulo $p$, ovvero $a_i\equiv a_j\pmod p$ e in particolare $p\mid\gcd(a_i-a_j,n)$ e abbiamo trovato un fattore di $n$.

    La cosa che si fa solitamente è calcolare la sequenza a due velocità diverse, e dunque fare $\gcd(x_{2i}-x_i,n)$ ad ogni step.

    Si può verificare che se $m\mid n$, allora vale $x_{2i}\equiv x_i\pmod m$ per $i$ circa dell'ordine di $O(\sqrt m)$.

    Dunque la complessità dell'algoritmo (assumendo $f$ random) è $O(\sqrt[4] n)$.


    \subsection{Pollard $p-1$}
    \begin{definition}
        Dato un bound $B$ si dice che $m$ è $B$-liscio se le potenze dei primi che dividono $n$ sono minori di $B$.
    \end{definition}

    Se $p\mid n$ e $p-1$ fosse $B$-liscio, preso $Q=\lcm(1,2,\dots,B)$ avremmo che $p-1\mid Q$.

    Ma allora $p\mid a^Q-1$ per ogni $a$, e in particolare per trovare $p$ si calcola $\gcd(a^Q-1,n)$.

    L'algoritmo fissa dunque $B,a$ e calcola per potenze successive $a^Q\mod n$, calcolando infine il $\gcd$.

    Purtroppo questo algoritmo ha molti point of failures: la scelta di $B$ influenza moltissimo, ma ha anche un grande peso computazionale.

    \subsection{Crivello quadratico}


    \section{Logaritmo discreto}

    \subsection{Baby step-giant step}

    \subsection{Pollard's $\rho$}

    \subsection{Index calculus}

    \section{Curve ellittiche}

    Posso anche usarle per fattorizzare e test di primalità, oltre a farci le potenze.


    \chapter{Codici}
    Un codice $C$ è semplicemente un insieme di parole formate da lettere di un certo alfabeto.

    A noi interessaranno però solo codici con una certa struttura; in particolare i nostri codici saranno sempre sottoinsiemi di un qualche $\F_q^n$.

    Quello che vogliamo fare è inviare messaggi con una certa ridondanza in modo che il ricevente possa accorgersi se sono stati effettuati errori di trasmissione ed eventualmente correggerli da sè (es: comunicare con le sonde in giro per lo spazio).

    \section{Distanze ed errori}
    Un errore è quando una lettera della parola ricevuta è diversa dalla corssipondente nella parola inviata.

    L'assunzione è che gli errori abbiano probabilità $<0.5$ e siano indipendenti: quando ci arriva una parola allora vogliamo correggerla con quella del codice con cui condivide più lettere (MLD).

    \begin{definition}
        Date $v=(a_1,\dots,a_n)$ e $w=(b_1,\dots, b_n)$ due parole, definiamo la \emph{distanza di Hamming} $d(v,w)=\#\{ i\st a_i\neq b_i \}$.

        Data una parola $v$, sia il suo \emph{peso} $\wt(v)=d(v,0)$.

        Dato un codice $C$ definiamo infine la \emph{distanza} del codice come $$d_C=\min_{\substack{v,w\in C\\v\neq w}}d(v,w)$$
    \end{definition}

    \begin{definition}
        Sia $v$ la parola inviata e $w$ la parola ricevuta. L'\emph{errore} è $e=w-v$.

        Diciamo che il codice $C$ \emph{rileva} l'errore $e$ se $v+e\not\in C\;\forall v\in C$, ovvero se $w$ non fa parte del codice.

        Diciamo inoltre che il codice \emph{corregge} $e$ con $v$ se vale $d(v',v+e)>d(v,v+e)$ per ogni $v\neq v'\in C$.
    \end{definition}

    Vediamo ora che la distanza è una quantità fondamentale di un codice:

    \begin{proposition}
        Sia $C$ un codice di distanza $d$. Allora
        \begin{itemize}
            \item il codice rileva tutti gli errori $e$ con $\wt(e)\le d-1$
            \item il codice corregge tutti gli errori con $\wt(e)\le\left\lfloor\frac{d-1}{2}\right\rfloor$
        \end{itemize}
    \end{proposition}

    Abbiamo dunque capito che sono molto importanti le palle centrate in parole del codice.

    \begin{oss}
        La palla $B(v,t)=\{w\in\F_q^n\st d(w,v)\le t \}$ ha cardinalità $$B_t=\sum_{i=0}^t \binom{n}{i}(q-1)^i$$
    \end{oss}

    \begin{proposition}[Hamming bound]
        Sia $C$ un codice di distanza $d$, e $t=\left\lfloor \frac{d-1}{2} \right\rfloor$. Allora vale $$B_t\cdot \#C\le q^n$$
    \end{proposition}
    \begin{definition}
        Se vale l'uguaglianza, diciamo che $C$ è un codice \emph{perfetto}, ovvero $\F_q^n$ viene partizionato completamente dalle palle di centro $v\in C$ e raggio $t$ (che è quello di cui sappiamo correggere).
    \end{definition}

    \section{Codici lineari}
    Dico che un codice $C$ è \emph{lineare} di \emph{dimensione} $m$ se è un sottospazio vettoriale $m$-dimensionale di $\F_q^n$.

    \begin{oss}
        Se $C$ è lineare, allora vale $d=\min_{v\neq0}\wt(v)$.
    \end{oss}

    Posso considerare allora una base $b_1,\dots, b_m$ di $C$, e la matrice $G$ che ha per righe i $b_i$. Ogni parola $v\in C$ è perciò della forma $uG$ con $u\in\F_q^m$.

    Osservo poi che $C$ è generato da $G$, ma anche da ogni matrice equivalente a $G$ con operazioni elementari di riga. In particolare posso prendere $G'=$, in modo che $uG'=(u\;\vline\; uX)$. Questa scelta di $G$ si dice \emph{codifica sistematica}, perché permette la decodifica immediata.

    \medskip
    Possiamo inoltre considerare $C^\perp$ lo spazio ortogonale a $C$, che avrà una base $w_1,\dots, w_{n-m}$, con matrice $H$ detta \emph{matrice di parità}.

    Vale infatti $GH^t=0$, ovvero $x\in C$ se e solo se $Hx=0$. Data una parola ricevuta $w$, chiamiamo \emph{sindrome} la quantità $Hw$, che ci dovrebbe dire dove e quali sono gli errori.

    Notiamo inoltre che se $G=\left(\begin{array}{c|c} I_m & X \end{array}\right)$, allora la matrice di parità corrispondente è $H=\left(\begin{array}{c|c} -X^t & I_{n-m} \end{array}\right)$.

    \begin{proposition}
        Un codice lineare ha distanza $d$ se e solo se $\rk H = d-1$, ovvero ogni $d-1$ righe sono indipendenti, ma esistono $d$ righe dipendenti.
    \end{proposition}

    Come correggo gli errori?

    Considero $\F_q^n/ C$ le classi laterali di $C$; osservo che le sindromi sono in corrispondenza biunivoca con le classi laterali.

    Se mi arriva una parola $w$, calcolo la sindrome $Hw$; tra tutte le parole in $w+C$ trovo quella con peso minore $v'$ (il cosiddetto \emph{coset leader}), e allora traduco $w$ con $w-v'$ che è la parola del codice più vicina a $w$.

    MA: ci possono essere più di un coset leader...

    \begin{proposition}[Singleton bound]
        Se $C$ è un codice lineare di tipo $(n,m,d)$ allora vale $d\le n-m+1$.
    \end{proposition}

    \begin{definition}
        Se un codice $C$ soddisfa $d=n-m+1$, viene detto MDS (maximum distance separable).
    \end{definition}

    \begin{theorem}[Gilbert-Varshamov]
        Siano $n,m,d$ fissati. Esiste un codice lineare di lunghezza $n$, dimensione $m$ e distanza $d$ su $\F_q$ se vale $$\sum_{i=0}^{d-2}\binom{n-1}{i}(q-1)^i<q^{n-m}$$

    \end{theorem}

    \subsection{Codice di Hamming binario}

    Fissato $r$, sia $H$ la matrice con $r$ righe e che ha per colonne le rappresentazioni binarie dei numeri $1,2,\dots,2^r-1$.

    Il codice di Hamming $\mathcal H_r$ è il codice binario di lunghezza $n=2^r-1$ che ha $H$ per matrice di parità.

    \begin{oss}
        Se $e_i$ è il vettore che ha zeri ovunque e un $1$ in posizione $i$, allora $He_i$ è la rappresentazione di $i$ in base $2$.
    \end{oss}

    \begin{proposition}
        Il codice $\mathcal H_r$ ha distanza $d=3$ ed è perfetto.
    \end{proposition}
    \begin{proof}
        La distanza si vede dall'indipendenza delle righe.

        Per verificare che è perfetto deve valere $b_1\cdot \# C=2^n$, ovvero $\left(\binom{n}{0}+\binom{n}{1}\right) 2^{n-r}=2^n$. Ma i binomiali valgono $1+n=2^r$, quindi l'uguaglianza è verificata.
    \end{proof}

    \subsection{Codice esteso}

    Se abbiamo un codice lineare con matrici $G,H$, possiamo considerare il codice $C'$ ottenuto aggiungendo un'ultima riga di check alla matrice $H$, ovvero $H'=\begin{pmatrix}
    H & 0\\ j & -1
    \end{pmatrix}$ dove $j$ è il vettore di tutti $1$ (stiamo quindi aggiungendo il check dello XOR). Osserviamo che $\rk H'=\rk H+1$.

    La matrice $G'$ corrispondente sarà della forma $G=(G\,\vline\, b)$; imponendo $G'(H')^t=0$ ricaviamo $b=Gj$, ovvero l'ultimo valore che aggiungiamo all'encoding è la somma di tutti i precedenti.

    L'utilità dei codici estesi è soprattutto su $\F_2$, poiché $\wt(v')=\wt(v)+1$ se $\wt(v)$ era dispari: allora possiamo trasformare un codice di distanza $d=2m-1$ in un codice di distanza $d'=2m$ al prezzo di un solo bit di lunghezza in più.

    \section{Codici ciclici}

    \begin{definition}
        Un codice $C$ di lunghezza $n$ è ciclico se $(a_1,\dots,a_n)\in C$ implica $(a_n,a_1,\dots,a_{n-1})\in C$.
    \end{definition}

    Definiamo poi la mappa $\varphi: C\to \faktor{\F_q[x]}{x^n-1}$ data da $\varphi((c_0,\dots,c_{n-1}))=c_0+c_1x+\dots+c_{n-1}x^{n-1}$.

    L'operazione di shift corrisponde alla moltiplicazione per $x$, quindi $\varphi(C)$ è un ideale dell'anello quoziente.

    Esiste quindi un polinomio $g(x)=g_0+g_1x+\dots+g_mx^m$ che genera l'ideale $C$, il che si traduce in:
    \begin{itemize}
        \item una parola $u(x)$ appartiene al codice se e solo se $g\mid u$ nel quoziente.
        \item il codice ha dimensione $k=n-m$, e una base è data da $g,xg,\dots, x^{k-1}g$.
    \end{itemize}

    Vale infine il teorema di caratterizzazione:
    \begin{theorem}
        Un polinomio $g$ è il generatore di un codice ciclico di lunghezza $n$ se e solo se $g(x)\mid x^n-1$.
    \end{theorem}


    Un modo di codificare è prendendo un polinomio $u(x)$ di grado $<k$, e codificandolo con $u(x)g(x)$.

    Questo corrisponde a prendere la matrice $G$ della base $g,xg,\dots,x^{k-1}g$, che si scrive tipo $$\begin{pmatrix}
    g_0 & \dots & g_m\\
    & g_0 & \dots & g_m\\
    & & \ddots & \ddots & \ddots \\
    & & & g_0 & \dots & g_m
    \end{pmatrix}$$

    Sia poi $g(x)h(x)=x^n-1$; questo $h$ dà esattamente la matrice di parità, poiché $f\in C$ se e solo se $h(x)f(x)\equiv0\pmod{x^n-1}$.

    \subsection{Codifica sistematica}
    Usiamo invece un altro modo di codificare: dato un messaggio $u$ di grado $<k$, considero la divisione con resto $x^{n-k}u(x)=a(x)g(x)+r(x)$. Invio allora $x^{n-k}u(x)-r(x)$, che è un elemento del codice. La decodifica avverrebbe dunque guardando le ultime $k$ cifre, mentre le prime $m$ sono di parità.

    Per scrivere esplicitamente le matrici devo considerare la base formata dagli $u=x^i$ per $i=0,\dots,k-1$, ovvero scrivere $x^{n-k+i}=a_i(x)g(x)+b_i(x)$, con $b_i(x)=\sum_{j=0}^{n-k-1}b_{i,j}x^j$; le matrici sono dunque

    $$G=\begin{pmatrix}
    -b_{0,0} & \dots & -b_{0,n-k-1} & 1 & 0 & \dots & 0\\
    -b_{1,0} & \dots & -b_{1,n-k-1} & 0 & 1 & \dots & 0\\
    \vdots & \ddots & \vdots & & & \ddots  \\
    -b_{k-1,0} & \dots & -b_{k-1,n-k-1} & 0 & 0 & \dots & 1
    \end{pmatrix}$$

    $$H=\begin{pmatrix}
    1 & 0 & \dots & 0 & b_{0,0} & \dots & b_{k-1,0} \\
    0 & 1 & \dots & 0 & b_{0,1} & \dots & b_{k-1,1} \\
     & & \ddots  & & \vdots &  \ddots  \\
    0 & 0 & \dots & 1 & b_{0,n-k-1} & \dots & b_{k-1,n-k-1}
    \end{pmatrix}$$

    Si può usare per l'encoding anche il polinomio $h$: sapendo che $hv=0$, dove $v$ è l'encoding di $u$, ci sono delle equazioni che permettono di ricavare $v_0,\dots,v_{n-k-1}$
     in funzione di $v_{n-k},\dots,v_{n-1}=u_0,\dots,u_{k-1}$.

     \medskip
     Infine per la decodifica la sindrome di $w$ è semplicemente $w\mod g$, e si procede poi a cercarne il coset leader.

    \subsection{Zeri di polinomi}
    Se $(n,q)=1$ allora il polinomio $x^n-1$ si spezza completamente su un $\F_{q^m}$, ovvero $x^n-1=\prod(x-\alpha^i)$ per qualche $\alpha$ radice primitiva dell'unità.

    In particolare si avrà $g(x)=(x-\alpha^{i_1})\cdots(x-\alpha^{i_{n-k}})$ per certi indici, quindi il codice ciclico generato da $g$ può essere visto come i polinomi $c\in\faktor{\F_q[x]}{x^n-1}$ tali che $c(\alpha^{i_j})=0$ per ogni $j$, ovvero la matrice di parità è data semplicemente dalla valutazione negli $\alpha^{i_j}$.


    \section{Codici BCH}

    \begin{lemma}[BCH bound]
        Sia $C$ un codice ciclico su $\F_q$ di lunghezza $n$ con generatore $g(x)$ tale che esistono $b\ge0,\delta\ge1$ per cui $g(\alpha^b)=\dots=g(\alpha^{b+\delta-2})=0$, dove $\alpha$ è un generatore di $\F_{q^r}^\ast$.\\
        Allora $C$ ha distanza almeno $\delta$.
    \end{lemma}

    \begin{definition}
        Si dice codice BCH binario di lunghezza $n$ e sistanza designata $\delta$ il codice ciclico con generatore $g=\lcm(m_b,\dots,m_{b+\delta-2})$ dove $m_b(x)$ è il polinomio minimo di $\alpha^i$ ($\alpha$ è un generatore di $\F_{2^r}$ con $n\mid 2^r-1$)
    \end{definition}

    Solitamente si usa $b=1, n=q^r-1$.

    \subsection{$2$-error correcting}
    Consideriamo $\F_{2^r}$ e un suo generatore $\beta$; sia $n=2^r-1$ e prendiamo $g(x)=\mu_\beta(x)\mu_{\beta^3}(x)\mid x^n-1$.

    Allora $g$ genera un codice ciclico di lunghezza $n$ e dimensione $n-2r$, la cui matrice di parità è data da $$\begin{pmatrix}
    1 & \beta & \beta^2 & \dots & \beta^i & \dots & \beta^{2^r-2}\\
    1 & \beta^3 & \beta^6 & \dots & \beta^{3i} & \dots & \beta^{3(2^r-2)}
    \end{pmatrix}$$

    La sindrome di una parola $w$ è esattamente $\begin{pmatrix}
    w(\beta) \\ w(\beta^3)
    \end{pmatrix}$, e da questa possiamo correggere fino a due errori, cioè fino a $e(x)=x^i+x^j$ (è un sistema di 2 equazioni in 2 incognite).

    \subsection{Decodifica}
    Se riceviamo una parola $w$, l'errore è $e=w-v$ e la sindrome è data da $s_i=w(\alpha^i)=e(\alpha^i)$. Assumiamo inoltre che $\deg e\le t$ dove $t$ è la capacità di correzione del codice.

    \begin{definition}
        Sia $L=\{ i\st e_i\neq0 \}$ l'insieme delle posizioni di errore.

        Il polinomio \emph{locatore d'errore} è $\sigma(x)=\prod_{\ell\in L}(1-x\alpha^\ell)$

        Il polinomio \emph{valutatore d'errore} è $\omega(x)=\sum_{\ell \in L}e_\ell\alpha^\ell \prod_{i\in L\setminus\{\ell\}}(1-x\alpha^i)$

        Il polinomio \emph{di sindrome} è $S(x)=e(\alpha)+e(\alpha^2)x+\dots+e(\alpha^{2t})x^{2t-1}$
    \end{definition}

    \begin{oss}
        I polinomi $\omega$ e $\sigma$ sono coprimi.
    \end{oss}

    Notiamo che c'è un errore in posizione $\ell$ se e solo se $\sigma(\alpha^{-\ell})=0$, nel qual caso troviamo l'errore tramite $e_l=-\frac{\omega(\alpha^\ell)}{\sigma'(\alpha^{-\ell})}$

    La decodifica si basa sul seguente risultato, detto \emph{equazione chiave}

    \begin{theorem}
        I polinomi $\sigma,\omega$ soddisfano $$\sigma(x)S(x)\equiv\omega(x)\pmod{x^{2t}}$$
        Inoltre sono gli unici (a meno di multipli) con $\deg \omega<\deg \sigma \le t$
    \end{theorem}

    Una volta calcolato il polinomio sindrome $S(x)$ bisogna dunque risolvere l'equazione chiave. Lo facciamo con l'algoritmo euclideo.

    \subsection{Codici Reed-Solomon}
    È un codice BCH con $n=q-1$; in particolare $g(x)=(x-\alpha^{b})\cdots(x-\alpha^{b+\delta-2})$ con $\alpha$ generatore di $\F_q$.

    \begin{oss}
        La distanza di un RS è esattamente $d=\delta$, quindi è MDS.
    \end{oss}

    Possiamo poi ricavare dal RS di distanza $d$ su $\F_{p^l}$ un codice su $\F_p$ vedendo ogni elemento di $\F_{p^l}$ come una parola lunga $l$ di $\F_p$: il nuovo codice ha dunque lunghezza $n'=ln=l(p^l-1)$.

    \section{Codici di Goppa}
    \begin{definition}
        Prendiamo un insieme $L=\{ \gamma_0,\dots,\gamma_{n-1} \}\subset \F_{q^r}$ e un polinomio $g(x)\in\F_{q^r}[x]$.

        $$\Gamma(L,g)=\left\{ (c_0,\dots,c_{n-1})\in\F_q^n\st \sum\frac{c_i}{x-\gamma_i}\equiv0\pmod{g(x)} \right\}$$
    \end{definition}

    \begin{oss}
        Con $g(x)=x^{\delta-1}$ e $L=\{ \alpha^{-i} \}$ otteniamo un codice BCH di distanza designata $\delta$.
    \end{oss}




    \chapter{Crittografia}








\end{document}
